%% LaTeX-Beamer template for KIT design
%% by Erik Burger, Christian Hammer
%% title picture by Klaus Krogmann
%%
%% version 2.4
%%
%% mostly compatible to KIT corporate design v2.0
%% http://intranet.kit.edu/gestaltungsrichtlinien.php
%%
%% Problems, bugs and comments to
%% burger@kit.edu

%% Class options
%%   aspect ratio options: 
%%   -- 16:9 (default)
%%   -- 4:3
%%   language options: 
%%   -- en (default)
%%   -- de
%%   position of navigation bar:
%%   -- navbarinline (default): bottom of the white canvas
%%   -- navbarinfooter : more compressed variant inside the footer
%%   -- navbarside : side bar at the left of the white canvas
%%   -- navbaroff : none
%% example: \documentclass[16:9,de,navbarinfooter]{sdqbeamer}
\documentclass[16:9,en,navbarinfooter]{sdqbeamer}

%% \documentclass{sdqbeamer} 

%% TITLE PICTURE

% if a custom picture is to be used on the title page, copy it into the 'logos'
% directory, in the line below, replace 'myimage' with the 
% filename (without extension) and uncomment the following line
% (picture proportions: 63 : 20 for standard, 169 : 40 for wide
% *.eps format if you use latex+dvips+ps2pdf, 
% *.jpg/*.png/*.pdf if you use pdflatex)

% \titleimage{myimage}

%% GROUP LOGO 

% for a custom group logo, copy your file into the 'logos'
% directory, insert the filename in the line below and uncomment it

\grouplogo{irl-logo}

% (*.eps format if you use latex+dvips+ps2pdf,
% *.jpg/*.png/*.pdf if you use pdflatex)

%% GROUP NAME

% for groups other than SDQ, please insert in the line below and uncomment it
% \groupname{My group}

% the presentation starts here 

\author{Caspar Friedrich Maximilian Nagy}

%% Title (and possibly subtitle) of the thesis
\title{Solving Real-World Robot Manipulation Tasks with Deep Reinforcement Learning}
\subtitle{Box Pushing on a real Panda Robot}

% Bibliography 
\usepackage{dsfont}
%\usepackage{multimedia}
%\usepackage{media9}
\usepackage{booktabs}
\usepackage{longtable}
\usepackage{color}
\usepackage{array}
\usepackage{algorithm}
\usepackage{algpseudocode}
\usepackage{pgfplots}
% \pgfplotsset{compat=1.15}
 \usepgfplotslibrary{groupplots}
\usepackage{subcaption}
\usepackage{pdflscape}
\usepackage{diagbox}
\usepackage{multicol}
\DeclareUnicodeCharacter{2212}{−}
\usepgfplotslibrary{groupplots,dateplot}
\usetikzlibrary{patterns,shapes.arrows}
\pgfplotsset{compat=newest}
\usepackage[citestyle=authoryear,bibstyle=numeric,hyperref,backend=biber%,style=verbose
]{biblatex}
\addbibresource{presentation.bib}
\bibhang1em
\usepackage{listings}
\begin{document}

%title page
\KITtitleframe{}

%table of contents

\begin{frame}
\frametitle{Agenda}
\tableofcontents
\end{frame}

\section{Motion Primitive-Based (Re-)Planning Policy (MP3)}
\begin{frame}{Motion Primitive-Based (Re-)Planning Policy (MP3)}

\begin{columns}[t]
    \begin{column}{\textwidth}
        \begin{itemize}
            \item Idea: let agent choose parameters of trajectory generator instead of low-level actions
                \begin{itemize}
                    \item We use ProDMPs for their replanning-ability
                \end{itemize}
            \item Leads to smooth trajectories, correlated exploration
            \item Can use sparse, non-markovian rewards
            \item Blackbox and replanning variant
            \item Train on-policy using Trust Region Projection Layers (TRPL)
        \end{itemize}
    \end{column}
    %\movie[width=\linewidth]{MP3-Replan Boxpushing}{media/7dof_boxpushing_replan.mp4}
    \begin{column}{0.5\textwidth}
    \end{column}
\end{columns}
\vspace{1cm}
     We have impressive results with simulated environments, but how well do they work in the real world?
\end{frame}

\begin{frame}{Box Pushing}
\section{Box Pushing}

\begin{columns}[t]
    \begin{column}{0.5\textwidth}
        \begin{itemize}
            \item Goal: Use a ``finger'' to push a box to a target pose
                \begin{itemize}
                    \item Random start pose
                    \item Fixed target pose
                \end{itemize}
        \end{itemize}
    \end{column}
    \begin{column}{0.5\textwidth}
        \begin{itemize}
                \item Challenges
                    \begin{itemize}
                            \item Underactuated system
                            \item Complex table-box interactions
                            \item Complex robot kinematic 
                            \item Trajectories must be safe and executable
                    \end{itemize}
        \end{itemize}
    \end{column}


\end{columns}
\center
    \includegraphics[height=3cm]{media/2dboxpushing.png}
    \vspace{.1cm}\\
    Box Pushing is a challenging benchmark problem for motion primitive reinforcement learning.
\end{frame}

\begin{frame}{Box Pushing}

\begin{columns}[t]
    \begin{column}{0.5\textwidth}
        \vspace{1cm}
        \begin{itemize}
            \item Goal: Use a finger to push a box to a target pose
                \begin{itemize}
                    \item Random start pose
                    \item Fixed target pose
                \end{itemize}
            \item Action space: 2D finger positions
            \item Observation:
                \begin{itemize}
                    \item Finger Position
                    \item Box Quaternion
                    \item Box Position
                \end{itemize}
        \end{itemize}
            \vspace{1em}
    \end{column}
    \begin{column}{0.5\textwidth}
        \[
        \begin{aligned}
    \text{Reward} &:= \text{Final Euclidean Distance} \\
             &+  \text{Final Rotational Distance} \\
             &+  \mathds{1} \left\{\text{success}\right\} \\
             &+  \max_t\  step(v_t) \\
             &+  \sum_t \left\Vert x_t - x_t^\text{clipped} \right\Vert \\
             \\
            \text{Success} &:= \max_t(v_t) < v_\text{max} \\
                &\land err_{pos} \leq 5 \text{cm} \\
                &\land err_{rot} \leq 0.5 \text{rad} \\
                \\
            \text{step}(v) &:= \alpha_1 v \quad &&(\text{if } v \leq v_\text{max})\\
                           &:= \alpha_2 v + \beta \quad &&(\text{else})\\
        \end{aligned}
    \]
    \end{column}
\end{columns}
\end{frame}

\section{Sim2Real}
\begin{frame}{Box Pushing --- Sim2Real}
\begin{columns}
    \begin{column}{0.2\textwidth}
        \begin{itemize}
                \item Our Approach: Sim2Real with Domain Randomization
                \item Box friction and mass as measured on the real box
                \item PD position control with randomized parameters 
        \end{itemize}
    \end{column}

    \begin{column}{0.4\textwidth}
\includegraphics[width=\linewidth]{media/ctrl_error.pdf}\\
    \end{column}

    \begin{column}{0.4\textwidth}
\includegraphics[width=\linewidth]{media/traj_error.pdf}
    \end{column}
\end{columns}

\end{frame}

\section{Lab Setup}
\begin{frame}{Box Pushing --- Lab Setup}
    \begin{columns}
    \begin{column}{0.45\textwidth}
    \vspace{1cm}
\includegraphics[width=\linewidth]{media/Architecture.pdf}

    \end{column}
    \begin{column}{0.45\textwidth}
    \vspace{1cm}
\includegraphics[width=\linewidth]{media/labsetup2.jpg}

    \end{column}
    \end{columns}

\end{frame}

\section{Results}
\begin{frame}{Results}
\center
    \includegraphics[width=.7\linewidth]{media/blackbox_presentation.pdf}\\
    \includegraphics[width=.7\linewidth]{media/replan_presentation.pdf}
\end{frame}


\appendix
\beginbackup{}
\backupend{}

\end{document}
